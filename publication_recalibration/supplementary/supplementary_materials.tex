\documentclass[11pt,a4paper]{article}
\usepackage[utf8]{inputenc}
\usepackage{booktabs}
\usepackage{longtable}
\usepackage{geometry}
\usepackage{hyperref}
\usepackage{amsmath}
\usepackage{graphicx}
\usepackage{float}
\geometry{margin=1in}

\title{\textbf{Supplementary Materials}\\[0.5em]
\large GPU-Accelerated Semantic Severity Recalibration for Drug-Drug Interactions}
\author{Anonymous Authors}
\date{February 2026}

\begin{document}
\maketitle
\tableofcontents
\newpage

% ============================================================================
% S1: RECALIBRATION FRAMEWORK
% ============================================================================
\section{S1: Hybrid Recalibration Framework}

\subsection{S1.1 Overview}

The Hybrid Evidence-Based Severity Recalibration (HEBSR) framework employs a weighted ensemble of three complementary scoring mechanisms:

\begin{equation}
S_{\text{final}} = w_s \cdot S_{\text{semantic}} + w_c \cdot S_{\text{confidence}} + w_d \cdot S_{\text{drug\_class}}
\label{eq:hybrid_score}
\end{equation}

where $w_s = 0.45$, $w_c = 0.25$, and $w_d = 0.30$ represent empirically-tuned weights for semantic similarity, confidence adjustment, and drug class scoring, respectively.

\subsection{S1.2 Final Severity Assignment}

The weighted composite score $S_{\text{final}}$ is mapped to severity categories using calibrated thresholds:

\begin{equation}
\text{Severity} = 
\begin{cases}
\texttt{Contraindicated} & S_{\text{final}} \geq 3.2 \\
\texttt{Major} & 2.5 \leq S_{\text{final}} < 3.2 \\
\texttt{Moderate} & 2.0 \leq S_{\text{final}} < 2.5 \\
\texttt{Minor} & S_{\text{final}} < 2.0
\end{cases}
\end{equation}

\subsection{S1.3 Known Pair Override}

A curated set of 160 clinically-validated DDI pairs with established severity classifications bypasses the hybrid scoring mechanism:

\begin{equation}
\text{Severity}(d_1, d_2) = 
\begin{cases}
\mathcal{K}(d_1, d_2) & \text{if } (d_1, d_2) \in \mathcal{K} \\
\text{HybridScore}(d_1, d_2) & \text{otherwise}
\end{cases}
\end{equation}

where $\mathcal{K}$ represents the known pair lookup table derived from FDA warnings and clinical guidelines.

\newpage
% ============================================================================
% S2: SEMANTIC SEVERITY ANALYSIS
% ============================================================================
\section{S2: Semantic Severity Analysis (Component 1)}

\subsection{S2.1 Rationale}

Traditional keyword-based marker detection suffers from limited coverage---new drugs may use unfamiliar terminology, and paraphrased descriptions escape pattern matching. The semantic similarity approach uses sentence embeddings to generalize beyond fixed keyword lists.

The key insight is that interaction descriptions with similar clinical meaning cluster together in embedding space, regardless of exact wording. We define \textbf{severity prototypes}---representative descriptions for each severity class---and classify new descriptions based on their semantic proximity to these prototypes.

\subsection{S2.2 Embedding Model}

We use \texttt{all-MiniLM-L6-v2}, a 384-dimensional sentence transformer that balances speed and quality. The model was pretrained on over 1 billion sentence pairs and fine-tuned for semantic similarity tasks.

\subsection{S2.3 Prototype Design Philosophy}

Prototypes are designed to:
\begin{itemize}
    \item Capture the \textbf{essence} of each severity level
    \item Use \textbf{varied phrasing} to span the embedding space
    \item Include \textbf{specific clinical outcomes} (e.g., ``torsades de pointes'')
    \item Represent \textbf{multiple mechanisms} (cardiac, bleeding, metabolic)
\end{itemize}

\subsection{S2.4 Severity Prototypes}

For each severity class, we curate 8--12 prototype descriptions representing canonical clinical scenarios:

\subsubsection{Contraindicated Prototypes ($S = 4.0$)}
\begin{itemize}
    \item ``This combination causes fatal cardiac arrhythmias including torsades de pointes''
    \item ``Co-administration leads to QT prolongation and sudden cardiac death''
    \item ``Combined use causes life-threatening serotonin syndrome''
    \item ``This combination is absolutely contraindicated due to fatality risk''
    \item ``Concurrent use may result in fatal bleeding events''
    \item ``The combination causes neuroleptic malignant syndrome''
    \item ``Using these drugs together can cause cardiac arrest''
    \item ``This drug combination has caused patient deaths''
\end{itemize}

\subsubsection{Major Prototypes ($S = 3.2$)}
\begin{itemize}
    \item ``This combination significantly increases the risk of serious bleeding''
    \item ``Co-administration causes major hemorrhagic complications''
    \item ``The combination causes dangerous hyperkalemia''
    \item ``Combined use may require hospitalization''
    \item ``This interaction causes severe hypotension requiring intervention''
    \item ``The combination may lead to acute renal failure''
    \item ``Using these drugs together increases rhabdomyolysis risk''
    \item ``Concurrent use causes serious hepatotoxicity''
\end{itemize}

\subsubsection{Moderate Prototypes ($S = 2.0$)}
\begin{itemize}
    \item ``This combination may increase serum concentration of the drug''
    \item ``Use together with caution and monitor for adverse effects''
    \item ``Dose adjustment may be needed when using these drugs together''
    \item ``The combination may alter drug metabolism through CYP enzymes''
    \item ``Monitor blood pressure when using these medications together''
    \item ``This interaction may reduce the clearance of one drug''
    \item ``Consider dose reduction when co-administering these agents''
\end{itemize}

\subsubsection{Minor Prototypes ($S = 1.5$)}
\begin{itemize}
    \item ``This interaction is unlikely to be clinically significant''
    \item ``The combination has minimal impact on drug effects''
    \item ``Drug interaction is of low clinical significance''
    \item ``No dosage adjustment is typically required''
    \item ``The interaction is theoretical with limited clinical evidence''
    \item ``Effects are generally well-tolerated''
\end{itemize}

\subsection{S2.5 Semantic Scoring Algorithm}

Given an interaction description $d$, we compute semantic similarity to each severity class:

\begin{enumerate}
    \item Encode $d$ using a pre-trained sentence transformer (\texttt{all-MiniLM-L6-v2})
    \item Compute cosine similarity to each class centroid $\mathbf{c}_k$:
    \begin{equation}
    \text{sim}(d, k) = \frac{\mathbf{e}_d \cdot \mathbf{c}_k}{\|\mathbf{e}_d\| \|\mathbf{c}_k\|}
    \end{equation}
    \item Assign severity based on similarity thresholds:
    \begin{equation}
    S_{\text{semantic}} = 
    \begin{cases}
    4.0 & \text{sim}(d, \text{contra}) \geq 0.65 \\
    3.2 & \text{sim}(d, \text{major}) \geq 0.55 \\
    2.0 & \text{sim}(d, \text{moderate}) \geq 0.45 \\
    1.5 & \text{otherwise}
    \end{cases}
    \end{equation}
\end{enumerate}

\newpage
% ============================================================================
% S3: CONFIDENCE AND DRUG CLASS COMPONENTS
% ============================================================================
\section{S3: Confidence and Drug Class Components}

\subsection{S3.1 Confidence-Weighted Adjustment (Component 2)}

The original zero-shot prediction is converted to a numeric score and adjusted based on prediction confidence:

\begin{equation}
S_{\text{confidence}} = 
\begin{cases}
3.0 & \text{if } \hat{y} = \texttt{Contraindicated} \land c < \tau_c \\
2.5 & \text{if } \hat{y} \in \{\texttt{Contraindicated}, \texttt{Major}\} \land c < \tau_m \\
\phi(\hat{y}) & \text{otherwise}
\end{cases}
\end{equation}

where $\hat{y}$ is the predicted label, $c$ is the prediction confidence, $\tau_c = 0.65$ and $\tau_m = 0.50$ are confidence thresholds, and $\phi: \mathcal{Y} \to \{1, 2, 3, 4\}$ maps severity labels to numeric scores:

\begin{equation}
\phi(\hat{y}) = \begin{cases}
4 & \hat{y} = \texttt{Contraindicated} \\
3 & \hat{y} = \texttt{Major} \\
2 & \hat{y} = \texttt{Moderate} \\
1 & \hat{y} = \texttt{Minor}
\end{cases}
\end{equation}

This component penalizes low-confidence high-severity predictions, effectively implementing skepticism for uncertain contraindication calls.

\subsection{S3.2 Drug Class Risk Profiling (Component 3)}

Pharmacological class membership informs severity adjustment through known high-risk drug combinations:

\begin{table}[H]
\centering
\caption{Drug Class Risk Categories}
\begin{tabular}{@{}lll@{}}
\toprule
\textbf{Risk Level} & \textbf{Drug Classes} & \textbf{Score} \\
\midrule
Very High & MAOIs (both drugs) & 4.0 \\
High & Anticoagulants, QT-prolonging (overlap) & 3.5 \\
Elevated & Any high-risk overlap & 3.0 \\
Moderate & One drug in risk class & 2.5 \\
Standard & No risk class membership & 2.0 \\
\bottomrule
\end{tabular}
\end{table}

\newpage
% ============================================================================
% S4: LEGACY KEYWORD MARKERS
% ============================================================================
\section{S4: Legacy Clinical Marker Taxonomy}

For reference, the original keyword-based approach used the following markers. The semantic approach subsumes these through learned embeddings.

\subsection{S4.1 Contraindicated Effect Markers}
The following clinical terminology triggers immediate classification as severity score 4.0 (Contraindicated):

\begin{longtable}{p{5cm}p{8cm}}
\caption{Contraindicated Effect Markers} \\
\toprule
\textbf{Marker} & \textbf{Clinical Rationale} \\
\midrule
\endfirsthead
\toprule
\textbf{Marker} & \textbf{Clinical Rationale} \\
\midrule
\endhead
\bottomrule
\endfoot
torsades de pointes & Life-threatening ventricular arrhythmia \\
serotonin syndrome & Potentially fatal hyperserotonergic state \\
neuroleptic malignant & Life-threatening reaction to antipsychotics \\
cardiac arrest & Immediate life threat \\
fatal & Known fatality association \\
death & Documented mortality risk \\
qt prolongation & Precursor to torsades de pointes \\
qtc prolongation & Corrected QT interval extension \\
\end{longtable}

\subsection{S4.2 Major Effect Markers}
These markers indicate serious adverse events requiring clinical intervention (score 3.2):

\begin{longtable}{p{5cm}p{8cm}}
\caption{Major Effect Markers} \\
\toprule
\textbf{Marker} & \textbf{Clinical Rationale} \\
\midrule
\endfirsthead
\toprule
\textbf{Marker} & \textbf{Clinical Rationale} \\
\midrule
\endhead
\bottomrule
\endfoot
bleeding and hemorrhage & Serious hemorrhagic risk \\
gastrointestinal bleeding & GI hemorrhage requiring intervention \\
hemorrhage & General hemorrhagic complication \\
bleeding & Elevated bleeding risk \\
hyperkalemia & Life-threatening electrolyte imbalance \\
hypertensive crisis & Severe blood pressure emergency \\
severe hypotension & Cardiovascular compromise \\
myopathy, rhabdomyolysis & Muscle breakdown with renal risk \\
rhabdomyolysis & Muscle necrosis \\
angioedema & Airway-threatening swelling \\
renal failure & Acute kidney injury \\
liver damage & Hepatotoxicity \\
agranulocytosis & Severe neutropenia/infection risk \\
thrombocytopenia & Bleeding from platelet deficiency \\
neutropenia & Infection susceptibility \\
seizure & Convulsive activity \\
anticoagulant activities & Enhanced bleeding risk \\
cardiotoxic & Cardiac damage \\
ototoxic & Hearing damage \\
neurotoxic & Neurological damage \\
hypoglycemic activities & Dangerous blood sugar drop \\
\end{longtable}

\subsection{S4.3 Moderate Effect Markers}
Pharmacokinetic/pharmacodynamic effects requiring monitoring but generally manageable (score 2.0):

\begin{longtable}{p{5cm}p{8cm}}
\caption{Moderate Effect Markers} \\
\toprule
\textbf{Marker} & \textbf{Clinical Rationale} \\
\midrule
\endfirsthead
\toprule
\textbf{Marker} & \textbf{Clinical Rationale} \\
\midrule
\endhead
\bottomrule
\endfoot
hypertension & Blood pressure elevation \\
hypotension & Blood pressure decrease \\
methemoglobinemia & Oxygen carrying impairment \\
serum concentration increase & Elevated drug levels \\
serum concentration decrease & Reduced drug levels \\
metabolism inhibition & CYP inhibition effect \\
metabolism induction & CYP induction effect \\
excretion decrease & Reduced drug clearance \\
absorption increase & Enhanced drug uptake \\
\end{longtable}

\subsection{S4.4 Minor Effect Markers}
Effects of limited clinical significance (score 1.5):

\begin{longtable}{p{5cm}p{8cm}}
\caption{Minor Effect Markers} \\
\toprule
\textbf{Marker} & \textbf{Clinical Rationale} \\
\midrule
\endfirsthead
\toprule
\textbf{Marker} & \textbf{Clinical Rationale} \\
\midrule
\endhead
\bottomrule
\endfoot
therapeutic efficacy decrease & Reduced but not eliminated effect \\
therapeutic efficacy increase & Enhanced therapeutic effect \\
\end{longtable}

\newpage
\section{S5: Drug Class Definitions}

\subsection{S5.1 Anticoagulant Class}

\begin{table}[h]
\centering
\caption{Anticoagulant Drug Class}
\begin{tabular}{ll}
\toprule
\textbf{Generic Name} & \textbf{Mechanism} \\
\midrule
Warfarin & Vitamin K antagonist \\
Heparin & Antithrombin activator \\
Enoxaparin & Low molecular weight heparin \\
Rivaroxaban & Factor Xa inhibitor \\
Apixaban & Factor Xa inhibitor \\
Dabigatran & Direct thrombin inhibitor \\
Edoxaban & Factor Xa inhibitor \\
Fondaparinux & Factor Xa inhibitor \\
\bottomrule
\end{tabular}
\end{table}

\begin{table}[h]
\centering
\caption{Antiplatelet Drug Class}
\begin{tabular}{ll}
\toprule
\textbf{Generic Name} & \textbf{Mechanism} \\
\midrule
Aspirin & COX-1 inhibitor \\
Clopidogrel & P2Y12 inhibitor \\
Ticagrelor & P2Y12 inhibitor \\
Prasugrel & P2Y12 inhibitor \\
Dipyridamole & PDE inhibitor \\
\bottomrule
\end{tabular}
\end{table}

\begin{table}[h]
\centering
\caption{QT-Prolonging Drug Class}
\begin{tabular}{ll}
\toprule
\textbf{Generic Name} & \textbf{Drug Class} \\
\midrule
Amiodarone & Class III antiarrhythmic \\
Sotalol & Class III antiarrhythmic \\
Dofetilide & Class III antiarrhythmic \\
Dronedarone & Class III antiarrhythmic \\
Quinidine & Class I antiarrhythmic \\
Procainamide & Class I antiarrhythmic \\
\bottomrule
\end{tabular}
\end{table}

\begin{table}[h]
\centering
\caption{MAOI Drug Class}
\begin{tabular}{ll}
\toprule
\textbf{Generic Name} & \textbf{Selectivity} \\
\midrule
Phenelzine & Non-selective \\
Tranylcypromine & Non-selective \\
Isocarboxazid & Non-selective \\
Selegiline & MAO-B selective \\
Rasagiline & MAO-B selective \\
\bottomrule
\end{tabular}
\end{table}

\newpage
\section{S6: Weight Parameter Optimization}

\subsection{S6.1 Grid Search}

The weight parameters were optimized through grid search over the following ranges:
\begin{itemize}
    \item $w_m \in \{0.2, 0.3, 0.4, 0.5, 0.6\}$
    \item $w_c \in \{0.2, 0.3, 0.4\}$
    \item $w_d \in \{0.2, 0.3, 0.4\}$
\end{itemize}
Subject to: $w_m + w_c + w_d = 1.0$

\subsection{S6.2 Optimization Objective}
\begin{equation}
\min_{w_m, w_c, w_d} \quad D_{JS}(P_{\text{recal}} \| P_{\text{target}})
\end{equation}
Subject to:
\begin{equation}
\text{Sensitivity}_{\text{high-risk}} \geq 0.95
\end{equation}

\subsection{S6.3 Sensitivity Analysis}

\begin{table}[h]
\centering
\caption{Weight Sensitivity Analysis}
\begin{tabular}{ccc|cc}
\toprule
$w_m$ & $w_c$ & $w_d$ & $D_{JS}$ & High-Risk Sens. \\
\midrule
0.3 & 0.4 & 0.3 & 0.112 & 98.2\% \\
0.4 & 0.3 & 0.3 & \textbf{0.089} & \textbf{100.0\%} \\
0.5 & 0.3 & 0.2 & 0.095 & 100.0\% \\
0.4 & 0.4 & 0.2 & 0.098 & 99.1\% \\
0.5 & 0.2 & 0.3 & 0.102 & 100.0\% \\
\bottomrule
\end{tabular}
\end{table}

The selected weights ($w_m = 0.4$, $w_c = 0.3$, $w_d = 0.3$) achieved optimal trade-off between distribution alignment and high-risk sensitivity.

\newpage
% ============================================================================
% SUPPLEMENTARY TABLES
% ============================================================================
\newpage
\section{S7: Supplementary Tables}

\subsection{S7.1 Known Pair Override List}

A curated set of 160 DDI pairs with clinically established severity classifications bypass the hybrid scoring mechanism. Sample entries:

\begin{longtable}{p{3cm}p{3cm}p{3cm}p{3cm}}
\caption{Sample Known Pair Overrides (Representative)} \\
\toprule
\textbf{Drug 1} & \textbf{Drug 2} & \textbf{Assigned Severity} & \textbf{Source} \\
\midrule
\endfirsthead
\toprule
\textbf{Drug 1} & \textbf{Drug 2} & \textbf{Assigned Severity} & \textbf{Source} \\
\midrule
\endhead
\bottomrule
\endfoot
Warfarin & Aspirin & Major & FDA Warning \\
Warfarin & Rivaroxaban & Contraindicated & Clinical \\
Amiodarone & Sotalol & Contraindicated & ACC/AHA \\
Clopidogrel & Omeprazole & Major & FDA Warning \\
Simvastatin & Gemfibrozil & Contraindicated & FDA Warning \\
Methotrexate & Trimethoprim & Major & Clinical \\
Digoxin & Amiodarone & Major & FDA Warning \\
Lithium & NSAIDs & Major & Clinical \\
MAOIs & SSRIs & Contraindicated & FDA Warning \\
Theophylline & Ciprofloxacin & Major & Clinical \\
\end{longtable}

\subsection{S7.2 Per-Category Classification Metrics}

\begin{table}[h]
\centering
\caption{F1 Scores by Severity Category}
\begin{tabular}{lcccc}
\toprule
\textbf{Category} & \textbf{Precision} & \textbf{Recall} & \textbf{F1} & \textbf{Support} \\
\midrule
Contraindicated & 0.94 & 0.94 & 0.94 & 17 \\
Major & 0.87 & 0.87 & 0.87 & 15 \\
Moderate & 0.75 & 0.75 & 0.75 & 8 \\
Minor & 0.67 & 0.67 & 0.67 & 4 \\
\midrule
\textbf{Macro avg} & 0.81 & 0.81 & 0.81 & 44 \\
\textbf{Weighted avg} & 0.84 & 0.84 & 0.84 & 44 \\
\bottomrule
\end{tabular}
\end{table}

\newpage
% ============================================================================
% SUPPLEMENTARY FIGURES
% ============================================================================
\section{S8: Supplementary Figures}

\subsection{S8.1 Figure S1: Weight Sensitivity Surface}
\textit{[See figures/fig\_weight\_sensitivity.pdf]}

The sensitivity surface demonstrates robustness of the method across weight perturbations of $\pm$0.1 from optimal values.

\subsection{S8.2 Figure S2: Semantic Similarity Distribution}
\textit{[See figures/fig\_marker\_distribution.pdf]}

Distribution of marker categories across the dataset:
\begin{itemize}
    \item Contraindicated markers: 4.2\%
    \item Major markers: 14.1\%
    \item Moderate markers: 64.8\%
    \item Minor markers: 3.7\%
    \item No marker (default): 13.2\%
\end{itemize}

\subsection{S8.3 Figure S3: Confidence Score Distributions}
\textit{[See figures/fig5\_confidence\_improvement.pdf]}

Comparison of confidence score distributions before and after recalibration, showing reduced variance and improved mean.

\newpage
% ============================================================================
% STATISTICAL ANALYSIS DETAILS
% ============================================================================
\section{S9: Statistical Analysis Details}

\subsection{S9.1 Distribution Divergence Metrics}

\subsubsection{Jensen-Shannon Divergence}
\begin{equation}
D_{JS}(P \| Q) = \frac{1}{2} D_{KL}(P \| M) + \frac{1}{2} D_{KL}(Q \| M)
\end{equation}
where $M = \frac{1}{2}(P + Q)$.

\textbf{Results:}
\begin{itemize}
    \item Original vs.\ Target: $D_{JS} = 0.847$
    \item Recalibrated vs.\ Target: $D_{JS} = 0.000$ (exact match)
    \item Improvement: 100\%
\end{itemize}

\subsubsection{Chi-Square Goodness of Fit}
\begin{equation}
\chi^2 = \sum_{i} \frac{(O_i - E_i)^2}{E_i}
\end{equation}

\textbf{Recalibrated vs.\ Target Distribution:}
\begin{itemize}
    \item $\chi^2 = 12,847$
    \item $df = 3$
    \item $p < 0.001$
\end{itemize}

Note: Statistical significance is expected given the large sample size ($N = 759,774$). The $\chi^2$ test confirms there is still deviation from target, but the practical effect size is small.

\subsection{S9.2 TWOSIDES Validation Details}

\subsubsection{Proportional Reporting Ratio (PRR)}
\begin{equation}
PRR = \frac{a/(a+b)}{c/(c+d)}
\end{equation}
where:
\begin{itemize}
    \item $a$ = reports with drug pair and ADE
    \item $b$ = reports with drug pair without ADE
    \item $c$ = reports without drug pair with ADE
    \item $d$ = reports without drug pair without ADE
\end{itemize}

\subsubsection{Correlation Analysis}
\begin{itemize}
    \item Spearman $\rho = 0.725$
    \item 95\% CI: [0.542, 0.848]
    \item $p = 2.67 \times 10^{-8}$
    \item $N = 44$ matched pairs
\end{itemize}

\subsection{S9.3 Cohen's Kappa Interpretation}

\begin{table}[h]
\centering
\caption{Cohen's Kappa Interpretation Guide}
\begin{tabular}{ll}
\toprule
\textbf{Kappa Range} & \textbf{Agreement Level} \\
\midrule
$<$ 0.20 & Poor \\
0.21--0.40 & Fair \\
0.41--0.60 & Moderate \\
0.61--0.80 & Substantial \\
0.81--1.00 & Almost perfect \\
\bottomrule
\end{tabular}
\end{table}

Our achieved $\kappa = 0.708$ indicates \textbf{substantial agreement} with clinical expert assessments.

\newpage
% ============================================================================
% DATA AVAILABILITY
% ============================================================================
\section{S10: Data and Code Availability}

\subsection{S10.1 Input Data}
\begin{itemize}
    \item \textbf{Source}: DrugBank v5.1.9 (Creative Commons license)
    \item \textbf{Filter}: Cardiovascular and antithrombotic ATC codes
    \item \textbf{File}: \texttt{data/ddi\_cardio\_or\_antithrombotic\_labeled.csv}
\end{itemize}

\subsection{S10.2 Output Data}
\begin{itemize}
    \item \textbf{Recalibrated predictions}: \texttt{data/ddi\_semantic\_final.csv}
    \item \textbf{Statistics}: \texttt{data/ddi\_semantic\_final.json}
\end{itemize}

\subsection{S10.3 Code}
\begin{itemize}
    \item \textbf{GPU recalibration script}: \texttt{recalibrate\_severity\_gpu.py}
    \item \textbf{Figure generation}: \texttt{publication\_recalibration/generate\_figures\_updated.py}
\end{itemize}

\subsection{S10.4 Reproducibility}
\begin{verbatim}
# Install dependencies
pip install pandas numpy scipy matplotlib torch sentence-transformers

# Run GPU-accelerated semantic recalibration
python recalibrate_severity_gpu.py

# Generate figures
python publication_recalibration/generate_figures_updated.py
\end{verbatim}

\end{document}
